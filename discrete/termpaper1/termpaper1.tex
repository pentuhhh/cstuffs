\documentclass[12pt, a4paper]{article}
\usepackage{amssymb}
\usepackage{amsmath}
\begin{document}

\title{Pre-Midterms Term Paper}
\author{Nikolai Tristan Pazon}

\begin{titlepage}
    \centering
    \vspace*{1cm}
    {\Huge \bfseries Mathematical Systems, Direct Proofs and Counter Examples \par}
    \vspace{1cm}
    {\large By: Nikolai Tristan Pazon \par}
    \vspace{1cm}
    {\large University of San Carlos \par}
    \vspace{1cm}
    {\large Bachelor of Science in Computer Science \par}
    \vspace{0.25cm}
    {\large Discrete Mathematics 2\par}
    

\end{titlepage}

\section{Introduction}

\hspace{0.5cm} In this paper we will explore and expound through ,mathematical proofing, several fundamental
 theorems and inequalities in Discrete Math. These proving challenges will task us to revisit and refine our basic understanding of Algebra, Number Theory and most of all writing proofs.

\vspace{1cm}
This paper requires the knowledge and skill of writing cohesive mathematical proofs by establishing 
a domain, hypothesis and a conclusion. Relying heavily on establishing statements that are supported by mathematical laws, theorems
%
and lemmas. Using these fundamental concepts, we are able to structure and justify our arguments or counter arguments
%
Our goal is to prove, disprove our hypothesis and possible create a better and more logically sound counter argument

\section{Euler's Number Proof}
\centering
Task: Prove that e, Euler's number is not a rational number 



\vspace{0.5cm}
{\fontsize{17.28pt}{18pt}\selectfont
\[
e = \sum_{k=0}^\infty \frac{1}{k!} \notin \mathbb{Q}
\]
}

\raggedleft

{\small Figure 1.0}

\centering
\vspace{0.5cm}



Recalling the definition of a rational number, denoted as $\mathbb{Q}$,

\vspace{0.5cm}
{\fontsize{17.28}{18pt}\selectfont
\[
\frac{a}{b} = \mathbb{Q}
\]
}

\raggedleft

{\small Figure 1.1}

\vspace{0.5cm}
\raggedright

Let us create a hypothesis, assume that euler's number is a rational number.
\newline

hypothesis: \centering {\fontsize{17.28}{18pt}\selectfont
\[e \in \mathbb{Q}
\]
}

\raggedright%

Suppose that the summation from k = 0 to $\infty$ can be expressed in a more simpler form

\centering{\fontsize{17.28pt}{18pt}\selectfont
\[ \sum_{k=0}^\infty \frac{1}{k!} = \frac{1}{0!} + \frac{1}{1!} + \frac{1}{2!} ... + \frac{1}{k!} = \frac{a}{b}
\]
}

\raggedleft

{\small Figure 1.2}

\centering{\fontsize{17.28pt}{18pt}\selectfont
\[ e = \frac{a}{b}
\]
}
\raggedleft

{\small Figure 1.2}

\vspace{0.5cm}

\raggedright

According to our hypothesis where $e \in \mathbb{Q}$, the reciprocal of eulers number
should also be rational

\centering{\fontsize{17.28pt}{18pt}\selectfont
\[ e = \frac{a}{b} \quad = \quad e^{x} = \frac{b}{a}
\]
} s.t. x = -1

\raggedleft
{\small Figure 1.3}

\centering{\fontsize{17.28pt}{18pt}\selectfont
\[ e^{-1} = \frac{1}{e} = \frac{b}{a} = \sum_{k=0}^\infty \frac{(-1)^{k}}{k!}
\]
} s.t. $a,b \in \mathbb{Z}$

\raggedleft
{\small Figure 1.4}

\raggedright

Using a, let us group the summation of the infinite series into two terms. The first term will be the summation from 0 to a and the second 
term will be the summation from a to infinity. 

\centering{\fontsize{17.28pt}{18pt}\selectfont
\[ e^{-1} = \sum_{k=0}^{a} \frac{(-1)^{k}}{k!} + \sum_{k = a + 1}^{\infty} \frac{(-1)^{k}}{k!}
\]
}

\raggedleft
{\small Figure 1.5}

\raggedright



\end{document}