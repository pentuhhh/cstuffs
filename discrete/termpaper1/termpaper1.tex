\documentclass[12pt, a4paper]{article}
\usepackage{amssymb}
\usepackage{amsmath}
\begin{document}

\title{Pre-Midterms Term Paper}
\author{Nikolai Tristan Pazon}

\begin{titlepage}
    \centering
    \vspace*{1cm}
    {\Huge \bfseries Mathematical Systems, Direct Proofs and Counter Examples \par}
    \vspace{1cm}
    {\large By: Nikolai Tristan Pazon \par}
    \vspace{1cm}
    {\large University of San Carlos \par}
    \vspace{1cm}
    {\large Bachelor of Science in Computer Science \par}
    \vspace{0.25cm}
    {\large Discrete Mathematics 2\par}
    

\end{titlepage}

\section{Introduction}

\hspace{0.5cm} In this paper we will explore and expound through ,mathematical proofing, several fundamental
 theorems and inequalities in Discrete Math. These proving challenges will task us to revisit and refine our basic understanding of Algebra, Number Theory and most of all writing proofs.

\vspace{1cm}
This paper requires the knowledge and skill of writing cohesive mathematical proofs by establishing 
a domain, hypothesis and a conclusion. Relying heavily on establishing statements that are supported by mathematical laws, theorems
%
and lemmas. Using these fundamental concepts, we are able to structure and justify our arguments or counter arguments
%
Our goal is to prove, disprove our hypothesis and possible create a better and more logically sound counter argument

\section{Euler's Number Proof}
\centering
Task: Prove that e, Euler's number is not a rational number 



\vspace{0.5cm}
{\fontsize{17.28pt}{18pt}\selectfont
\[
e = \sum_{k=0}^\infty \frac{1}{k!} \notin \mathbb{Q}
\]
}

\raggedleft

{\small Figure 1.0}

\centering
\vspace{0.5cm}



Recalling the definition of a rational number, denoted as $\mathbb{Q}$,

\vspace{0.5cm}
{\fontsize{17.28}{18pt}\selectfont
\[
\frac{a}{b} = \mathbb{Q}
\]
}

\raggedleft

{\small Figure 1.1}

\vspace{0.5cm}
\raggedright

Let us create a hypothesis, assume that euler's number is a rational number.
\newline

hypothesis: \centering {\fontsize{17.28}{18pt}\selectfont
\[e \in \mathbb{Q}
\]
}

\raggedright%

Suppose that the summation from k = 0 to $\infty$ can be expressed in a more simpler form

\centering{\fontsize{17.28pt}{18pt}\selectfont
\[ \sum_{k=0}^\infty \frac{1}{k!} = \frac{1}{0!} + \frac{1}{1!} + \frac{1}{2!} ... + \frac{1}{k!} = \frac{a}{b}
\]
}

\raggedleft

{\small Figure 1.2}

\centering{\fontsize{17.28pt}{18pt}\selectfont
\[ e = \frac{a}{b}
\]
}
\raggedleft

{\small Figure 1.2}

\vspace{0.5cm}

\raggedright

According to our hypothesis where $e \in \mathbb{Q}$, the reciprocal of eulers number
should also be rational

\centering{\fontsize{17.28pt}{18pt}\selectfont
\[ e = \frac{a}{b} \quad = \quad e^{x} = \frac{b}{a}
\]
} s.t. x = -1

\raggedleft
{\small Figure 1.3}

\centering{\fontsize{17.28pt}{18pt}\selectfont
\[ e^{-1} = \frac{1}{e} = \frac{b}{a} = \sum_{k=0}^\infty \frac{(-1)^{k}}{k!}
\]
} s.t. $a,b \in \mathbb{Z}$

\raggedleft
{\small Figure 1.4}

\raggedright

Using a, let us group the summation of the infinite series into two terms. The first term will be the summation from 0 to a and the second 
term will be the summation from a to infinity. 

\centering{\fontsize{17.28pt}{18pt}\selectfont
\[ e^{-1} = \frac{b}{a}= \sum_{k=0}^{a} \frac{(-1)^{k}}{k!} + \sum_{k = a + 1}^{\infty} \frac{(-1)^{k}}{k!}
\]
}

\raggedleft
{\small Figure 1.5}

\centering{\fontsize{17.28pt}{18pt}\selectfont
\[ \frac{b}{a} - \sum_{k = 0}^{a} \frac{(-1)^k}{k!} = \sum_{k = a + 1}^{\infty} \frac{(-1)^{k}}{k!}
\]
}
\raggedleft
{\small Figure 1.6}

\raggedright

Multiply each term with \[(-1)^{(a+1)}a!\]

\vspace{0.5cm}
\centering{\fontsize{17.28pt}{18pt}\selectfont

term 1:

\[
    (-1)^{(a+1)} b(a-1)! 
    \]
}
\raggedright

this term is an integer since all constants, variables and integers. any operation between integers that integers
excluding division and root, will always result an integer

\centering{\fontsize{17.28pt}{18pt}\selectfont

term 2:

\[
    (-1)^{(a+1)} (\frac{a!}{2!} \dots + \frac{(-1)^{a}a!}{a})
    \]
}

\raggedright

in this series, every denominator is always less than the numerator, it will always result
an integer since the denominator is a factor of the numerator

\centering{\fontsize{17.28pt}{18pt}\selectfont
\[
    \forall a, b \in \mathbb{Z}, a > b  
\]
}
\centering{\fontsize{17.28pt}{18pt}\selectfont
\[
    \frac{a!}{b!} = c, s.t c \in \mathbb{Z}
\]

}


\vspace{1cm}
\raggedright

since term 1 and term 2 is a subtraction between two integers, we can conclude that the left
side of the equation is an integer

\centering{\fontsize{17.28pt}{18pt}\selectfont

\[
  \mathbb{Z} = \sum_{k = a + 1}^{\infty} \frac{(-1)^{k}}{k!}
\]

}

\vspace{1cm}
\raggedright

Let us expound on the term on the right side of the equation

\centering{\fontsize{17.28pt}{18pt}\selectfont

\[
  =  \frac{1}{a + 1} - \frac{1}{(a + 1)(a + 2)} + \dots
\]

}

\vspace{1cm}
\raggedright

in this expression, an upper and lower bound is defined. The first term is the 
upper bound. The succeeding terms are the lower bound such that it is lower than the first term
as k reaches infinity, the summation of the succeding terms will approach zero. Therefore this
this expression will result in a number that is irrational

\centering{\fontsize{17.28pt}{18pt}\selectfont

\[
    \frac{1}{a + 1} - \frac{1}{(a + 1)(a + 2)} + \dots \notin \mathbb{Z}
\]

}
\newpage
\raggedright
\section{Minskowski's Inequality for Sums}

Task: Prove Minskow's Inequality for Sums

\centering{\fontsize{17.28}{18pt}\selectfont

\[
  [\sum_{k = 1}^{n} |a_k + b_k|^{p}]^\frac{1}{p} \leq [\sum_{k = 1}^{n} |a_k|^{p}]^\frac{1}{p}
  + [\sum_{k = 1}^{n} |b_k|^{p}]^\frac{1}{p}
\]

\[
    \forall a, b, p \in \mathbb{R} 
\]
\[
    p  > 1, (a_k, b_k) > 0
\]  
}

\vspace{1cm}
\raggedright
Let us directly prove Minskowski's Inequality for sums by the
triangle Inequality theorem

\vspace{1cm}
\centering{\fontsize{17.28}{18pt}\selectfont

Triangle Inequality Theorem
\[
    |a + b| \leq |a| + |b|
\]

}

\raggedright
Let us first raise terms on both sides to P

\centering{\fontsize{17.28pt}{18pt}\selectfont

\[
    |a + b|^{p} \leq |a|^{p} + |b|^{p}  
\]
}

\raggedright

Do summation notation for all k from 1 to n

\centering{\fontsize{17.28pt}{18pt}\selectfont

\[
  \sum_{k = 1}^{n} |a_k + b_k|^{p} \leq \sum_{k=1}^{n} |a_k|^{p} + \sum_{k =1 }^{n} |b_k|^{p} 
\]

}

\raggedright

raise to 1 over p by property of absolute value

\centering{\fontsize{17.28pt}{18pt}\selectfont

\[
  [\sum_{k = 1}^{n}|a_k + b_k|^{p}]^{\frac{1}{p}} \leq [\sum_{k = 1}^{n} |a_k|^{p}]^{\frac{1}{p}} + [\sum_{k = 1}^{n}|b_k|^{p}]^{\frac{1}{p}}
\]

}



\end{document}