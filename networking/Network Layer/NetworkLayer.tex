\documentclass{article}
\usepackage{array}
\usepackage{geometry}
\geometry{a4paper, portrait, margin=0.5in}
\begin{document}

\section{\textbf{Network Layer Characteristics}}

The network layer provides services to allow end devices to exchange data
THe principle network layer communication protocols are the IP version 4 and version 6
In this layer, there are four basic operations.

\begin{itemize}
    \item Addressing End devices
    \item Encapsulation
    \item Routing
    \item De-Encapsulation
    
\end{itemize}

\section{\textbf{Ip Encapsulation}}

Ip encapsulates the transport layer segment and can either use an IPv4 or IPv6
packet and will not have an impact on the layer 4 segment. IP packet will be examined by all
layer 3 devices as it traverses the network. THe IP Addressing does not change from source to 
destination. The NAT introduces changes in the addressing but it is for another topic

\section{\textbf{Characteristics of IP}}

IP is meant to have low overhead and may be described as 

\begin{itemize}
    \item Connectionless
    \item Best Effort
    \item Media Independent
\end{itemize}

\section{\textbf{Connectionless}}

Recalling back to the Characteristic of Ip as Connectionless\dots

\begin{itemize}
    \item Ip does not establish a connection with the destination before sending the packet.
    \item There is no control information needed such as synchronizations and acknowledgements
    \item The destination will receive the packet when it arrives, but no pre-notifications are sent by Ip.
    \item IF there is a need for connection oriented traffic, then another protocol will handle this, typically TCP 
    
\end{itemize}

\section{\textbf{Best Effort}}

Recalling back to the Characteristic of IP as Best Effort\dots

\begin{itemize}
    \item IP will not guranteed delivery
    \item IP has reduced overhead since there is no mechanism to resend unreceived data
    \item It does not expect acknowledgements
    \item there is no way to know if the other device is online or has received the packet
\end{itemize}

\section{\textbf{Media Independent}}

Recalling back to the Characteristic of IP as Media Independent\dots

\vspace{0.5cm}
Ip is unreliable since:

\begin{itemize}
    \item It cannot manage or fix undelivered or corrupt packets
    \item cannot retransmit automatically after an error
    \item cannot realign out of sequence packets
    \item must rely on other protocols for these functions
\end{itemize}


IP is media independent since:

\begin{itemize}
    \item it does not concern itself with the type of fram required at the data link layer or the media
    type at the physical layer
    \item IP can be sent over any media type
\end{itemize}

\begin{flushleft}

    The nework layer will establish teh Maximum Transmission Unit. It will receive it from the control information
    sent by the data link layer. The network then establishes the MTU size. 

    Fragmentation is when Layer 3 splits the IPv4 packet into smaller units, while IPv6 does not. This introdces
    more latency. An example of this is when router goes from Ethernet to a slow WAN with a smaller MTU; MTU mismatch
\end{flushleft}


\section{\textbf{IPv4 Packet Header}}

\begin{flushleft}

    IPv4 is the primary communication protocol for the network layer. The network has many purposes
    \begin{itemize}
        \item ensures the packet is sent in the correct direction
        \item contains information for network layer processing in various fields
        \item the information in the header is used by all layer 3 devices that handle the packet
    \end{itemize}

\end{flushleft}

\section{\textbf{IPv4 Packet Header Fields}}

\begin{flushleft}
    The IPv4 network header is in binary, contains several fields of information
    the diagram is read from left to right, 4 bytes per each line. The two most important fields
    are the source and destination

    \textbf{Significant fields in the IPv4 header}

    \begin{center}
        \begin{tabular}{ |m{5cm}|m{5cm}|m{2cm}|}
            \hline

            Function & Description & Size \\

            \hline

            Version & This will be for V4 as opposed to V6, a 4 bit field = 0100 & $\frac{1}{2}$ Byte\\
            \hline

            Differentiated Services & Used for QoS: DiffServ - DS field or the older interServe Type of Service & 2 Bytes\\
            \hline

            header Checksum & Detect corruption n the IPv4 Header & 2 Bytes\\
            \hline

            Time to Live & Layer 3 hop count. When it becomes 0 the router will discard the packet & 1 Byte\\
            \hline

            Protocol & I.D.'s next level protocol: ICMP, TCP, UDP, etc. & 1 Byte\\
            \hline

            Source IPv4 Address & 32bit source address & 4 Bytes\\
            \hline

            destination IPv4 Address & 32bit destination address & 4 Bytes\\
            \hline
        \end{tabular}
    \end{center}
\end{flushleft}

\section{\textbf{IPv6 Packets - Limitations of IPv4}}

\begin{flushleft}
    IPv4 has three major limitations\dots
    
    \begin{itemize}
        \item IPv4 has address depletion, this means that we are running ot of IPv4 addresses. There are 4 billion
        possible addresses, while there are already 8 billion people in the world that may have 1 or more device that will need
        \item it lacks end to end connectivity, this is the reason why private addressing and NAT was created.
        \item Increases network complexity, NAT was made as a short term fix. NAT creates issues on the network as a side effect, causing latency and troubleshooting issues
    \end{itemize}
\end{flushleft}

\section{\textbf{IPv6 Overview}}

\begin{flushleft}
    IPv6 was developed by Internet Engineering Task force in order to overcome the limitations of IPv4. IPv6 has
    increased address space from 32bit to 128bits. It also has improved packet handling with a simplified header and
    fewer fields. It also eliminates the need for NAT since there is a hge amount of addressing, there is no need
    to use private addressing internally
\end{flushleft}

\section{\textbf{IPv4 vs IPv6 packet headers}}

\begin{flushleft}

    \begin{itemize}
        \item IPv6 header is simplified but not smaller
        \item The header is fixed at 40 bytes or octets long
        \item Several IPv4 fields were removed such as the flag, frament Offeset and Header Checksum
    \end{itemize}
\end{flushleft}

\textbf{Significant fields in the IPv6 Header}

\begin{center}
    \begin{tabular}{ |m{5cm}|m{5cm}|m{2cm}| }
        \hline
        
        Function & Description & Size \\
        \hline

        Version & This will be for v6 as opposed to v4, a 4 bit field = 0110 & $\frac{1}{2}$ Byte \\
        \hline

        Traffic Class & Used for QoS: Equivalent to Diffserv & 1 Byte\\
        \hline

        Flow Label & Informs device to handle identical flow labels the same way, 20 bit field & $2\frac{1}{2}$ Byte\\
        \hline

        payload length & This 16-bit field indicates the length of the data portion of the IPv6 Packet & 2 Byte\\
        \hline
        
        Next Header & I.D.'s the next level protocol: ICMP, TCP, UDP, etc. & 1 Byte\\
        \hline

        Hop Limit & Replaces TTL field layer 3 hop count & 1 Byte\\
        \hline
        
        Source IPv4 Address & 128 bit source address & 4 Byte\\
        \hline

        Destination IPv4 Address & 128 bit destination address & 4 Bytes\\
        \hline

    \end{tabular}
\end{center}

\begin{flushleft}
    IPv6 packet may also contain extension headers. These provide optional network layer information
    it is placed betweein the IPv6 header and the payload. This may be used for fragementation, security, mobility
    support and etc. Routers will not fragment IPv6 packets
\end{flushleft}

\section{\textbf{Host Forwarding Decision}}

\begin{flushleft}
    Packets are always created at the source. Each host devices creates their own routing table.

    A host can send packets to the following:
        \begin{itemize}
            \item Itself - 127.0.0.1 (IPv4), ::1 (IPv6)
            \item Local Hosts - destination is on the same LAN
            \item Remote Hosts - devices are not on the same LAN
        \end{itemize}

    The source device determines whether the destination is local or Remote. It determines it 
    using its own IP address and subnet mask and the destination IP address. For IPv6 the source
    uses the network address and prefix advertised by the local router.

    Local traffic is then dumped out the host interface to be handled by an intermediary device. Remote
    traffic is forwarded directly to the default gateway on the lan.
\end{flushleft}

\section{\textbf{Default Gateway}}

\begin{flushleft}
    A router or layer 3 switch can be a defalt-gateway. A default gateway must have an IP address in the same range
    as the rest of the Lan. It can accept data from the LAN and is capable of forwarding traffic off the LAN. and
    most importantly, it can route to other networks. 

    If a device has no defalt gateway or a bad gateway, its traffic will not be able to leave the LAN
\end{flushleft}

\section{\textbf{A host routes to the default gateway}}

\begin{flushleft}
    The host will know the default gateway (DGW) either statically or through DHCP in IPv4.
    IPv6 sends the DGW through a router solicitation or can be configured manually. A DGW is a static 
    route in the routing table. All devices on the LAN will need the DGW of the router if they intend to send traffic.
\end{flushleft}

\section{\textbf{Host Routing Tables}}

\begin{flushleft}
    On windows, rote print or netstat -r to display the routing table of the PC

    Three sections displayed by these two commands

    Interface List - all potential interface and MAC Addressing
    
    IPv4 Routing table

    IPv6 Routing table
\end{flushleft}


\section{\textbf{Router Packet Forwarding Decision}}

\begin{flushleft}
    When a router receives the frame from a host device\dots

    \begin{itemize}
        \item Packet arriveson the Gigabit Ethernet 0/0/0 interface of roter R1, which then the layer 2 ethernet header and trailer is de-encapsulated
        \item R1 examines the destination IPv4 address of the packet and searches for the best match in its IPv4 routing table.
        \item The route entry indicates that this packet is to be forwarded to R2
        \item R1 encapsulates the packet into an new ethernet header and trailer, forwards to the next hop router- R2
    \end{itemize}
\end{flushleft}


\section{\textbf{IP Router Routing Table}}

\begin{flushleft}
    There are three types of routes in a router's routing table:

    Directly connected - These routes are automatically added by the router, provided the interface is active and has Addressing

    Remote - These are the routes the router does not have a direct connection and may be learned Manually with a static route or Dynamically
    by using a routing protocol to have the routers share information with each other

    Default Route - This forwards all traffic to a specific direction when there is not a match in the routing table
\end{flushleft}

\section{\textbf{Static Routing}}

\begin{flushleft}
    Static routes must be configured manually. It must also be adjusted manually by the administrator when there is a change 
    in the topology. This is meant for small non-redundant networks. Often used in conjunction with a dynamic routing protocol for
    configuring a default rote
\end{flushleft}

\section{\textbf{Dynamic Routing}}

\begin{flushleft}
    Dynamic Routes Automaticall Discover remote networks. They also maintain up-to-date information. It chooses the best path
    to the destination. It also find new best paths when there is a change in the topology
\end{flushleft}

\section{\textbf{Introdction to an IPv4 Routing Table}}
\begin{flushleft}
    The \textbf{show IP route} command shows the following route sources:

    \begin{itemize}
        \item \textbf{L} - Directly connected local interface IP address
        \item \textbf{C} - Directly Connected Network
        \item \textbf{S} - Static route was manually configred by an administrator
        \item \textbf{O} - OSPF
        \item \textbf{D} - EIGRP
    \end{itemize}

    This command shows types of routes:
    \begin{itemize}
        \item Directly Connected - C and L
        \item Remote Rotes - O, D, etc.
        \item Default Routes - S*
    \end{itemize}
\end{flushleft}

\section{\textbf{Ethernet Frame Deciphering}}
\begin{flushleft}
    Identify the Ethernet Header (14 bytes):

    \begin{itemize}
        \item The first 6 bytes represent the Destination MAC Address.
        \item The next 6 bytes represent the Source MAC Address.
        \item The last 2 bytes represent the EtherType, which indicates the type of the payload.

    \end{itemize}
        
    Identify the IP Header (20 bytes for IPv4):

    \begin{itemize}
        \item The first byte represents the Version and IHL (Internet Header Length).
        \item The next byte represents the DSCP (Differentiated Services Code Point) and ECN (Explicit Congestion Notification).
        \item The next 2 bytes represent the Total Length of the IP packet.
        \item The next 2 bytes represent the Identification.
        \item The next 2 bytes represent the Flags and Fragment Offset.
        \item The next byte represents the TTL (Time To Live).
        \item The next byte represents the Protocol.
        \item The next 2 bytes represent the Header Checksum.
        \item The next 4 bytes represent the Source IP Address.
        \item The last 4 bytes represent the Destination IP Address.
    \end{itemize}

    Identify the TCP Header (20 bytes minimum)

    \begin{itemize}
        \item The first 2 bytes represent the Source Port.
        \item The next 2 bytes represent the Destination Port.
        \item The next 4 bytes represent the Sequence Number.
        \item The next 4 bytes represent the Acknowledgment Number.
        \item The next 2 bytes represent the Data Offset, Reserved, and Flags.
        \item The next 2 bytes represent the Window Size.
        \item The next 2 bytes represent the Checksum.
        \item The next 2 bytes represent the Urgent Pointer.
        \item The remaining bytes represent the Options (if any).
    \end{itemize}

    pneumonics

    6 6 2 1 1 2 2 2 1 1 2 4 4


        

\end{flushleft}
\end{document}